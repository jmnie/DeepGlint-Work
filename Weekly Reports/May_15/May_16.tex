\documentclass[11pt, a4paper]{article}
\usepackage{graphicx, fullpage, hyperref, listings}
\usepackage{appendix, pdfpages, color}
\usepackage{indentfirst} %段首空两格 棒
\usepackage{chngpage} 
\usepackage{tocloft}            % This squashes the Table of Contents a bit
\usepackage{pdfpages}
\usepackage{multirow}
\usepackage{amsmath}
\usepackage{amssymb}
\usepackage{framed}
\usepackage[UTF8]{ctex}
\usepackage{array}%需要该宏包



\setlength\cftbeforesecskip{3pt}
\renewcommand{\contentsname}{\centerline{\textbf{Content}}}
\graphicspath{{images/}}

\usepackage{multicol}

\usepackage{graphicx}
\usepackage{epstopdf}
\hypersetup{CJKbookmarks,%
	bookmarksnumbered,%
	colorlinks,%
	linkcolor=black,%
	citecolor=black,%
	plainpages=false,%
	pdfstartview=FitH}

%%%%%%%代码语法高亮设置

\usepackage{color}

\definecolor{pblue}{rgb}{0.13,0.13,1}
\definecolor{pgreen}{rgb}{0,0.5,0}
\definecolor{pred}{rgb}{0.9,0,0}
\definecolor{pgrey}{rgb}{0.46,0.45,0.48}

\usepackage{listings}
\lstset{
	language=Java,
	showspaces=false,
	showtabs=false,
	%%%%%
	frame = single,
	stepnumber = 2,  
	numbersep = 4pt, 
	 numbers=left,
	%breakatwhitespace=false, 
	tabsize=2,  
	%%%%%
	breaklines=true,
	showstringspaces=false,
    breakatwhitespace=false, 
	commentstyle=\color{pgreen},
	keywordstyle=\color{pblue},
	stringstyle=\color{pred},
	basicstyle=\ttfamily,
	%moredelim=[il][\textcolor{pgrey}]{$$},
	%moredelim=[is][\textcolor{pgrey}]{\%\%}{\%\%},
}


%%%%%%%%代码语法高亮设置

\definecolor{MyLightYellow}{cmyk}{0,0.,0.2,0} 

\setlength{\parskip}{4pt}        % sets spacing between paragraphs
\interfootnotelinepenalty=500    % this prevents footnotes breaking across pages

\title{\includegraphics[width=0.45\textwidth]{dg}
        \\Future Work Plan Commencing From May 16 \\ 面部表情识别的工作计划 }          % <<<<<<<<< change the title as appropriate
\author{Jiaming Nie}                    % <<<<<<<<< module code

\begin{document}
\begin{titlepage}
	
%\date{\today}
\maketitle
\addtocontents{toc}{\protect\thispagestyle{empty}} % because we don't want a page number on the title page
% Thanks to Huang Shanyue for suggesting this 




%\date{\today}
\thispagestyle{empty}  %去除首页页码

\end{titlepage}

%\tableofcontents
%\listoffigures

\newpage

\section{使用不同的网络训练}

\subsection{fer2013数据集}
Fer2013的数据集中共有7种表情,训练集共有28710个样本,测试集有3590个样本,验证集共有3590个样本。

计划使用的网络模型: vgg16, vgg19, ResNet. 
\subsubsection{详细时间表}

\begin{table}[htbp] 
	\begin{center}
		\caption{具体时间表}
		\begin{tabular}{|l|p{320pt}|} \hline
		May 16, 周三 & 熟悉DGNet的基本操作,模型搭建 \\ \hline
		May 17, 周四 & 在DGNet上对fer2013进行训练(也可以周三进行) \\ \hline
		May 18,周五 & 总结并继续EmotionNet论文的阅读 \\ \hline
		\end{tabular}
		\label{tab:timetable}
	\end{center}
\end{table}	

\subsection{AffectNet数据集}

AffectNet数据集还在申请中,尚未得到回复。预计下周可能可以获取数据。初步的计划与对fer2013相同,用三种不同的网络对AffectNet数据集进行训练。

如果5月21日或之前能拿到数据,具体的时间规划表如下:

\subsubsection{训练的时间表}

\begin{table}[htbp] 
	\begin{center}
		\caption{具体时间表}
		\begin{tabular}{|l|p{320pt}|} \hline
			May 21 - May 22, 周一 - 周二 & 熟悉并了解AffectNet数据集,看数据类型是可以直接提取还是需要切割人脸 \\ \hline
			May 23 - 24, 周三 - 周四 & 微调已有的模型,进行训练 \\ \hline
			May 25, 周五 & 继续阅读EmotionNet论文,深入理解 \\ \hline
		\end{tabular}
		\label{tab:table-2}
	\end{center}
\end{table}	


\section{EmotionNet算法的实现}

EmotionNet通过对人脸提取特特征点,进一步处理提取特征值来进行对不同人脸进行分类(基于贝叶斯方法)。提取脸部特征点需要其他paper中所实现的特定函数。

实现过程计划为两种:

\begin{itemize}
\item[1.] 提取特征向量后使用不同的CNN深度模型进行训练,获得准确率
\item[2.] 复现论文中的方法,进行训练,获取准确率
\end{itemize}

\subsubsection{时间表}

\begin{table}[htbp] 
	\begin{center}
		\caption{EmotionNet算法}
		\begin{tabular}{|l|p{320pt}|} \hline
			May 28 - May 30, 周一 - 周三 & 对Fer2013数据集进行人脸特征点提取以及特征向量的制作\\ \hline
			May 31 - June 1, 周四-周五 & 理解论文中所使用的贝叶斯模型并进行训练  \\ \hline
			6.1 之后的下一周 & 对AffectNet的数据集进行特征点提取,重复fer2013的做法 \\ \hline
		\end{tabular}
		\label{tab:table-3}
	\end{center}
\end{table}	

备注:如果AffectNet的数据集不能顺利收集,考虑提前进行EmotionNet算法对fer2013数据集的复现以及其他数据集的验证。
%\subsection{对于fer2013 数据集}

%\subsection{对于AffectNet数据集}





\bibliographystyle{IEEEtran}  
%\bibliography{MyRefs} 
%\addcontentsline{toc}{section}{References}





%-------------------------------------------------------------------------------------------------------





\end{document}
